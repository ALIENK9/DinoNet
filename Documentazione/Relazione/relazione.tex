\documentclass[12pt]{article}
\usepackage[utf8]{inputenc}
\usepackage{graphicx}
\usepackage[hidelinks]{hyperref}
\usepackage{fancyhdr}
\usepackage{lastpage}
\usepackage{url}
\PassOptionsToPackage{hyphens}{url}\usepackage{hyperref}
%\usepackage{geometry}
%\usepackage{makeidx}
%\geometry{left=2.5cm,right=2.5cm,top=2.5cm,bottom=2.5cm}
%\makeindex

%Per Fancy
\newcommand{\cornice}[1] %come parametro richiede il nome del file
{
	\pagestyle{fancy}
	\renewcommand{\footrulewidth}{0.4pt}% default is 0pt
	\fancyhf{}
	\fancyhead[L]{\nouppercase{\leftmark}}
	\fancyhead[R]{breakingbug@gmail.com}
	\fancyfoot[L]{#1}
	\fancyfoot[R]{Pagina \thepage\ di \pageref{LastPage}}
	\setlength{\headheight}{13.6pt} % come chiede fancyhdr
}


\renewcommand*\contentsname{Sommario}

\title{\textbf{Dino Jaws}\\\large Relazione del progetto di Tecnologie Web a.a. 2017-18}
\author{Alessandro Zangari, Cristiano Tessarolo, Matteo Rizzo}
\date{\today}

\begin{document}
	\maketitle
	\tableofcontents
	\clearpage
	\cornice{Relazione del progetto di Tecnologie Web}
	
	\section{Descrizione dei requisiti}
	Lo scopo di questa attività di progetto è creare un sito Web che riporti informazioni e articoli sul tema dei dinosauri. I contenuti informativi del sito sono di due tipi: le \textbf{schede descrittive}, una per ciascun dinosauro, e gli \textbf{articoli} su scoperte e teorie che riguardano il mondo preistorico. Il sito è particolarmente rivolto ad un pubblico giovane, in particolare vogliamo favorire la fruibilità dei contenuti a ragazzi di età nell'intorno dei 10 anni. Nel sito è inoltre possibile registrarsi per lasciare commenti agli articoli.
	
	\subsection{Frontend}
	
	\subsubsection{Parte pubblica}
	Descriviamo brevemente le diverse sezioni, ciascuna corrisposta da un'apposita voce del menù:
	\begin{itemize}
		\item \textbf{Home}: contiene una sezione introduttiva che spiega in che tipo di sito ci si trova e che tipo di informazioni contiene. È inoltre presente una barra di ricerca che permette di cercare fra i contenuti del sito ed un sezione con i contenuti del giorno, articoli o schede descrittive;
		\item \textbf{Storia}: una sezione introduttiva che ripercorre la storia dei dinosauri durante ciascun periodo geologico in cui sono vissuti;
		\item \textbf{Specie}: qui vengono messi in evidenza i contenuti giornalieri (una scheda ed un articolo), ed i link alle schede degli ultimi dinosauri aggiunti al database. È anche possibile visualizzare la lista completa delle schede descrittive di dinosauro, tramite il link sotto la barra di ricerca. Ogni scheda descrive un solo dinosauro e contiene un'immagine, la descrizione dell'animale e talvolta qualche curiosità;
		\item \textbf{Articoli}: accedendo a questa pagina viene visualizzato immediatamente l'articolo del giorno, e sotto la raccolta degli ultimi articoli aggiunti. Analogamente alla sezione \textit{Specie}, è possibile accedere alla lista completa degli articoli. Ogni articolo ha la sua sezione di commenti lasciati dagli utenti registrati al sito;
		\item \textbf{Accedi}: da questa area ogni utente può registrarsi o accedere al proprio account per scrivere o modificare i propri commenti agli articoli.
		Anche gli amministratori effettuano l'accesso da questa pagina.
	\end{itemize}

	\subsubsection{Parte privata}
	È la sezione accessibile solamente agli utenti amministratori. Questi utenti, possono aggiungere contenuti (articoli o schede dinosauro), e rimuovere commenti degli utenti. Naturalmente possono anch'essi inserire commenti.
	
	\subsection{Backend}
	...
	
	
	
	\section{Usabilità}
	Considerato l'aspettativa di utenza, abbiamo ritenuto opportuno focalizzarci sulla semplicità dei contenuti e di utilizzo del sito, ma cercando di ottenere un risultato grafico accattivante. Abbiamo fatto largo uso di immagini, perché efficaci ad attirare l'attenzione ed a intrattenere il nostro tipo di pubblico.\\
	Abbiamo tenuto in considerazione il fatto che spesso ragazzi e bambini usufruiscono di contenuti web da tablet, più che da desktop. Per questo abbiamo scelto di rendere la navigazione da tablet equivalente a quella da desktop. Le immagini decorative sono quindi visualizzate anche su schermi da 1024px, che è la tipica dimensione di uno schermo di un tablet.
	La presenza di molte immagini e il layout grafico "a piastrelle" che abbiamo scelto, ha anche imposto l'utilizzo dello scroll, soprattutto da mobile. Abbiamo valutato attentamente la bontà di questa scelta, ben sapendo che scroll troppo lunghi possono peggiorare l'esperienza utente. Tuttavia abbiamo considerato che oggi moltissimi contenuti su mobile, anche non web, sono presentati con layout simili e fanno ampio utilizzo di scorrimento verticale. Pertanto, riteniamo che, in particolare gli utenti giovani, siano familiari con questa organizzazione dei contenuti. \\
	Per quanto riguarda l'utilizzo da smartphone, abbiamo tenuto in considerazione che la navigazione avviene spesso da connessione dati cellulare. Al fine di non provocare un consumo massiccio di dati per la fruizione del sito, abbiamo scelto di non visualizzare diversi elementi puramente decorativi. Inoltre, visto che schermi molto piccoli visualizzerebbero ogni elemento uno sotto l'altro, e non più di uno affiancati, si elimina anche il problema di scorrimenti interminabili per reperire un contenuto in fondo alla pagina.\\
	
	\subsection{Media queries}
	Abbiamo fatto uso di media queries per garantire la compatibilità con tutti i dispositivi. Alcuni problemi sono sorti con i dispositivi Apple, che non visualizzavano correttamente l'effetto parallasse inserito nella sezione \textit{Storia}. Per questo abbiamo deciso di disabilitare questo effetto su ogni dispositivo mobile di tale marca, indipendentemente dalla dimensione.\\
	Il foglio di stile \textit{print.css} definisce un layout adatto per la stampa. Tutte le immagini decorative sono rimosse, il font viene inoltre impostato su \textit{Times} e quello colorato ritorna di colore nero.\\
	Il contenuto del giorno e la sezione \textit{Ultime aggiunte} non vengono mostrati nel layout di stampa, questo perché supponiamo che l'utente sia interessato alla stampa del solo contenuto principale della pagina, quindi il testo degli articoli o delle schede. Tuttavia, nel caso in cui si voglia stampare la pagina \textit{Home} oppure la pagina principale delle sezioni \textit{Articoli} o \textit{Specie}, che sono delle raccolte di contenuti che introducono le relative sezioni, allora i contenuti suggeriti saranno stampati, uno sotto l'altro.
	
		
	\section{Accessibilità}
	\subsection{Introduzione}
	...
	\subsection{Test}
	Per verificare l'accessibilità, tutte le pagine del sito sono state testate con i seguenti tool:
	\begin{itemize}
		\item \textbf{Total Validator}: utilizzato anche per la validazione delle pagine e dei fogli di stile, e disponibile in versione gratuita al sito: 
		\url{http://www.totalvalidator.com/validator/Validator};
		\item \textbf{FireFox Web Developer Toolbar}: un plugin per il browser Firefox;
		\item \textbf{Fangs - Screen Reader Emulator}: disponibile al sito: \url{http://sourceforge.net/projects/fangs};
		\item \textbf{IE Accessibility Toolbar}: compatibile solo con internet Explorer ed installabile al sito\\ \url{http://webaccessibile.org/articoli/la-barra-dellaccessibilita-versione-20};
		\item \textbf{Cynthia Says}: disponibile al sito \url{http://www.contentquality.com}.
	\end{itemize}

	\subsection{Risultati dei test}
	...
	
	\section{Compatibilità e dipendenze}
	\subsection{Compatibilità con i browser}
	
	\subsection{JavaScript}
	Abbiamo utilizzato JavaScript per rifinire alcuni aspetti grafici. Tuttavia JavaScript non è necessario per un corretto utilizzo del sito. Il più grosso limite è l'impossibilità di aprire il menù laterale da mobile, e cliccando il pulsante del menù si verrà rimandati (con un ancora) al fondo della pagina dove è posizionata una replica del menù, con qualche effetto in meno (non segna la scheda attiva). Da desktop, invece, si percepisce ancora meno la necessità di JavaScript.
	
\end{document}